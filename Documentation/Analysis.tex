\documentclass[11pt]{article}
\usepackage[top=1in, bottom=1in, left=1in, right=1in]{geometry}
\usepackage{listings}
\usepackage{color}

\renewcommand{\baselinestretch}{1.4}
\setlength{\parskip}{0.8em}
 
\definecolor{codegreen}{rgb}{0,0.6,0}
\definecolor{codegray}{rgb}{0.5,0.5,0.5}
\definecolor{codepurple}{rgb}{0.58,0,0.82}
\definecolor{backcolour}{rgb}{0.95,0.95,0.92}

\lstdefinestyle{codestyle}{
	language=Java,
    backgroundcolor=\color{backcolour},   
    commentstyle=\color{codegreen},
    keywordstyle=\color{magenta},
    numberstyle=\tiny\color{codegray},
    stringstyle=\color{codepurple},
    basicstyle=\footnotesize\ttfamily,
    breakatwhitespace=false,         
    breaklines=true,                 
    captionpos=b,                    
    keepspaces=true,                 
    numbers=left,                    
    numbersep=5pt,                  
    showspaces=false,                
    showstringspaces=false,
    showtabs=false,                  
    tabsize=2
}
\lstset{style=codestyle}

\newcommand{\quotes}[1]{``#1"}

\begin{document}

\begin{center}
\textbf{ECE 458: Engineering Software For Maintainability \\
Senior Design Course\\
Spring 2015\\[0.2in]}
Evolution 1 Analysis\\
Brian Bolze, Jeff Day, Henrique Rusca, Wes Koorbusch
\end{center}

\tableofcontents

% Writen Analysis (25%) Along with every software deliverable, you will turn in a written docu- ment which will cover two main points:

% 1. A retrospective on how your previous design choices impacted your work to meet the current set of requirements. This section should analyze not only where your good design choices made things easy, but also where your bad design choices made things hard. For both of these points, you should analyze how/why these design choices helped or hindered you. For bad design choices, you should discuss what you might have done differently in the past to avoid the problem this time around.

% 2. An evaluation of your current design, with an analysis of its strengths and weaknesses going forwards. This section should justify your current design choices, explaining why you think they will be beneficial to you in the long run. If you recognize weaknesses in your current design, you should discuss them—including an explanation of why they are there, and how you plan to fix them in future submissions.

% These documents should not only deep analysis of the strengths and weaknesses of your design choice, but also be well written. Ideally, the retrospective section of submission N would connect back to the forward-looking analysis of submission N-1 (i.e., Did things you think would be beneficial actually end up helping you? Did the weaknesses you identified come back to bite you? Did you fix your weaknesses this time around?).

\pagebreak

\section{Introduction}

Good software design is often seen not only as a science but as an art. It is a craft and, like any other type of engineering, is only mastered over time. The fundamentals of good software design, however, remain sound. In this senior design course, we plan on synthesizing our four years of knowledge through the development of a robust and long-lasting software application implementing a web-based calendar. We plan on applying the core principles of good design to our code and to our design process, while continuously evaluating, refining, and improving on our skills of the software engineering trade. 

For our project, we built an application using the Ruby on Rails framework. This framework aided us in developing highly modular and reusable code due to the MVC architecture and Rails' powerful web application stack. With the additional help of great documentation and a strong community, we were able to develop a functional product within weeks, despite minimal domain expertise. While our current application has noticeable areas for improvement, our robust model, well thought out API, and extensive front-end templates promise a maintainable foundation for future development. 

\section{Project Plan}

\subsection{Design Goals}

In order to set ourselves up for a successful project, it was really important to us that we lay out our design goals and priorities before moving forward. We aimed at grounding our design discussions in the fundamentals, which we hoped would help us set up a solid foundation for a project of such a significant size. The core design principles that we focused on primarily were modularity (Open-Closed Principle), re-usability, and the DRY principle. Another consideration that we put a lot of thought into was our language and framework choice, which will be expanded on in the next section. Besides promoting good code design, a well-chosen framework would allow us to get up to speed quickly, which was another important criteria for our team given our experience level with web applications. During our initial stages, a particular focus was placed on developing a robust, accurate, and extensible model for our data. This would lay the groundwork for our API with the front-end components, which we saw as vital to our success down the line. 


\subsection{Language Choice}

% RoR vs Django

% RoR design philosophy
%%% SHY
%%% MVC
%%% Convention over Configuration

%% Modular -- independence --> work by ourselves
%% DRY --> keep business logic in the model, filters in controller
%% Reusable --> Partials in View

% Ruby
%%% Dynamically typed
%%% embedded ruby...

The language decision was extremely important for our group for a few reasons. Firstly, no one in our group had previous experience in developing web applications. Accordingly, we stressed finding a language and framework with a strong community, good documentation, and abundant resources. Secondly, seeing as our project would be maintained over an entire semester, we strived to find an environment that would support and promote good design practices. To do this, we would need a language with strong object orientation and a framework with a highly modular architecture. Lastly, we also needed the flexibility to implement our application with high customizability to meet any obscure requirements. After extensive research and testing, we found that Ruby on Rails was able to meet our needs better than any other framework. 

% originally started with Django, chiefly because two members of the team had a strong background in Python, but did not turn out to be much of an advantage
% switched over due to roadblocks getting off the ground
% Realized rails had larger community, better documentation, and Rails is a widely sought out skill so we wanted to learn about it


\subsection{High-Level API}

% PitL vs PitS
stuf stuf stuf

\section{Design Review}

\subsection{Status}

stuf stuf stuf

\subsection{Design}

stuf stuf stuf

\subsection{Alternate Designs}

stuf stuf stuf

\section{Next Steps}

stuf stuf stuf


\end{document}